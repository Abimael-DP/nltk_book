%\documentclass{beamer}             % for slides
\documentclass[handout]{beamer}    % for handout
\mode<presentation>
{
  \usetheme{Pittsburgh}
  \setbeamercovered{transparent}
  \beamerdefaultoverlayspecification{<+->}
}

\mode<handout>
{
  \usetheme{default}
  \usecolortheme{default}
  \useoutertheme{default}
  \usepackage{pgf}
  \usepackage{pgfpages}
  \pgfpagesuselayout{4 on 1}[a4paper,landscape,scale=0.9]
  \setjobnamebeamerversion{handout.beamer}
}

\mode<article>
{
  \usepackage{fullpage}
  \usepackage{pgf}
  \usepackage{hyperref}
  \setjobnamebeamerversion{notes.beamer}
}

\usepackage[english]{babel}
\usepackage[latin1]{inputenc}
\usepackage{times}
\usepackage[T1]{fontenc}

\date{\today}

\subject{Natural Language Toolkit}



\title{Introduction to Computational Linguistics\\LSA 325}

\author{Steven Bird \and Ewan Klein \and Edward Loper}
\institute{
  University of Melbourne, AUSTRALIA
  \and
  University of Edinburgh, UK
  \and
  University of Pennsylvania, USA
}

%%%%%%%%%%%%%%%%%%%%%%%%%%%%%%%%%%%%%%%%%%%%%%%%%%%%%%%%%%%%%%%%%%%%%%%%%%%%%%%%%%
%%%%%%%%%%%%%%%%%%%%%%%%%%%%%%%%%%%%%%%%%%%%%%%%%%%%%%%%%%%%%%%%%%%%%%%%%%%%%%%%%%

\begin{document}


%%%%%%%%%%%%%%%%%%%%%%%%%%%%%%%%%%%%%%%%%%%%%%%%%%%%%%%%%%%%%%%%%%%%%%%%%%%%%%%%%%

\begin{frame}
  \titlepage
\end{frame}


\begin{frame}
  \frametitle{Materials and Resources}
  \begin{itemize}
  \item Chapters from the NLTK Book
    (\url{http://nltk.org/index.php/Book})
    \begin{itemize}
      \item Initially reusing readings for LSA 110 (Chapters 1--5)
      \item Second installment (Chapters 6 -- 13) should be available
        at Stanford Bookstore towards end of next week.
    \end{itemize}

    \item \textbf{Su07-LSA-325} page on \url{http://coursework.stanford.edu}
    \item Main NLTK page: \url{http://nltk.org}
      \begin{itemize}
        \item Chatroom
        \item \texttt{nltk-users} Mailing List
      \end{itemize}

   \end{itemize}
\end{frame}

\begin{frame}
  \frametitle{Assumptions}
  \begin{itemize}
  \item Knowledge of Python corresponding to material covered in
    Chapter 2
    \begin{itemize}
    \item If you haven't finished working through Chapter 2, please do
      so by Monday 9 July
    \end{itemize}

  \item A working installation of NLTK, including corpora
  \end{itemize}

\end{frame}


\begin{frame}
  \frametitle{Syllabus}
  \begin{description}
  \item[Class 1, Thu 5 July] Intro, Words (Ch.\ 3)
  \item[Class 2, Mon 9 July] Tagging (Ch.\ 4)
  \item[Class 3, Thu 12 July] Chunking (Ch.\ 5)
  \item[Class 4, Mon 16 July] Linguistic Data Management, Grammars \&
    Parsing (Chs.\ 13, 5)
  \item[Class 5, Thu 19 July] More Parsing, Feature-based Grammar
    (Chs.\ 8, 9)
  \item[Class 6, Mon 23 July] Compositional Semantics (Ch.\ 11)
  \item[Class 7, Thu 26 July] Language Engineering (Ch.\ 12)
 
   \end{description}
\end{frame}

\begin{frame}
  \frametitle{Assessment}
  \begin{itemize}
    \item Certain exercises from the book will be designated as required for credit.
    \item Submit via Dropbox on \texttt{coursework.stanford.edu}
    \item In addition: 2--3 pages critique of some aspect of the book,
      with concrete suggestions for fixes or improvement.
 
  \end{itemize}
\end{frame}

\begin{frame}
  \frametitle{Data Sets}
  \begin{tabular}{ll}
    `theoretical' grammars & a few hundred words\\
    Brown corpus           & 1 million words\\
    British National Corpus & 100 million words \\
    LDC Gigaword Corpus    & 1,000 million words \\
  \end{tabular}
\end{frame}

\begin{frame}
  \frametitle{Textual Inference (or RTE)}
  \begin{description}
  \item[Text] Soprano's Square: Milan, Italy, home of the famed La
    Scala opera house, honored soprano Maria Callas on Wednesday when
    it renamed a new square after the diva.
  \item[Hypothesis] La Scala opera house is located in Milan, Italy.
  \end{description}
\end{frame}



% \begin{frame}
%   \frametitle{Python and NLTK}
%   \begin{itemize}
%     \item Pre-session for \textit{Introduction to
%         Computational Linguistics} (LSA 325)
%     \item First steps in using Python and Natural Language Toolkit (NLTK)
%     \item Why Python?
%       \begin{itemize}
%       \item designed to be easy to learn;
%       \item good for processing linguistic data;
%       \item good for interactive experiments.
%       \end{itemize}
%     \item Many online tutorials (see \url{www.python.org})
%   \end{itemize}
% \end{frame}


% \begin{frame}
%   \frametitle{Materials and Resources}
%   \begin{itemize}
%   \item Chapter 2, \textit{Programming Fundamentals and Python} in the NLTK Book
%     (\url{http://nltk.org/index.php/Book})
%     \item \textbf{Su07-LSA-110} page on \url{http://coursework.stanford.edu}
%     \item Main NLTK page: \url{http://nltk.org}
%       \begin{itemize}
%         \item Chatroom
%         \item \texttt{nltk-users} Mailing List
%       \end{itemize}

%    \end{itemize}
% \end{frame}

% \begin{frame}
%   \frametitle{Audience and Goals}
%   \begin{itemize}
%   \item We are assuming you have not done programming before.
%   \item So, getting you to a point where:
%     \begin{itemize}
%     \item you have got some confidence in using basic Python commands;
%     \item you can use Python for carrying out simple operations on text;
%     \item you can do all the easy and intermediate exercises in
%       Chapter 2;
%     \item you have found out where to get more information (fellow
%       students, the web, textbooks)
%     \end{itemize}
%    \end{itemize}
% \end{frame}

% \begin{frame}
%   \frametitle{Syllabus}
%   \begin{description}
%   \item[Class 1] Manipulating strings, lists and other sequences.
%   \item[Class 2] Conditionals, dictionaries, functions and regular
%     expressions.
%   \item[Class 3] Preview of NLTK chapters on Words and Tagging
%    \end{description}
% \end{frame}

% \begin{frame}
%   \frametitle{Almost there \ldots}
%   \begin{itemize}
%   \item Installation CDs
%   \item Today: at least Python
%   \item Tomorrow: full NLTK installation
%   \item Homework: catch up on exercises and reading
%    \end{itemize}
% \end{frame}




\end{document}
