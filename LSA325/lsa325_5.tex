\documentclass[t]{beamer}             % for slides
%\documentclass[handout]{beamer}    % for handout
\mode<presentation>
{
  \usetheme{Pittsburgh}
  \setbeamercovered{transparent}
  \beamerdefaultoverlayspecification{<+->}
}

\mode<handout>
{
  \usetheme{default}
  \usecolortheme{default}
  \useoutertheme{default}
  \usepackage{pgf}
  \usepackage{pgfpages}
  \pgfpagesuselayout{4 on 1}[a4paper,landscape,scale=0.9]
  \setjobnamebeamerversion{handout.beamer}
}

\mode<article>
{
  \usepackage{fullpage}
  \usepackage{pgf}
  \usepackage{hyperref}
  \setjobnamebeamerversion{notes.beamer}
}

\usepackage[english]{babel}
\usepackage[latin1]{inputenc}
\usepackage{times}
\usepackage[T1]{fontenc}

\date{\today}

\subject{Natural Language Toolkit}



\title{Introduction to Computational Linguistics\\LSA 325}

\author{Steven Bird \and Ewan Klein \and Edward Loper}
\institute{
  University of Melbourne, AUSTRALIA
  \and
  University of Edinburgh, UK
  \and
  University of Pennsylvania, USA
}

%%%%%%%%%%%%%%%%%%%%%%%%%%%%%%%%%%%%%%%%%%%%%%%%%%%%%%%%%%%%%%%%%%%%%%%%%%%%%%%%%%
%%%%%%%%%%%%%%%%%%%%%%%%%%%%%%%%%%%%%%%%%%%%%%%%%%%%%%%%%%%%%%%%%%%%%%%%%%%%%%%%%%

\begin{document}


%%%%%%%%%%%%%%%%%%%%%%%%%%%%%%%%%%%%%%%%%%%%%%%%%%%%%%%%%%%%%%%%%%%%%%%%%%%%%%%%%%

\begin{frame}
  \titlepage
\end{frame}


\begin{frame}

  \frametitle{Compositional Semantics}

  \begin{itemize}
  \item Contrast with lexical semantics
  \item Meaning of a phrase is a function of the meaning of its parts
  \item Truth-conditions: minimum hurdle for a theory of meaning
  \item Entailment ($\phi \Rightarrow \psi$:  every situation that makes
    $\phi$ true also make $\psi$ true
 

  \end{itemize}
\end{frame}

\begin{frame}

  \begin{exampleblock}{Entailment Examples}
      \begin{itemize}
    \item \textit{Kim eats toasted bagels} $\Rightarrow$ \textit{Kim eats
        bagels}
    \item \textit{Lee sings and dances} $\Rightarrow$ \textit{Lee sings}
    \item \textit{Lee sings songs to Kim} $\Rightarrow$ \textit{Lee sings
        songs to someone}
    \item \textit{Kim hates all green vegetables and calabrese is a green
      vegetable} $\Rightarrow$ \textit{Kim hates
       calabrese}
    \end{itemize}
  \end{exampleblock}

\end{frame}

\begin{frame}

  \frametitle{Truth in a model, version 1}

  \begin{itemize}
  \item A model is a pair $\langle D, V\rangle$
  \item $V:$ Individual terms $\mapsto$ entities in $D$
  \item $V:$ 1-place predicates $\mapsto$ sets of entities
  \item $V:$ 2-place predicates (relations) $\mapsto$ sets of pairs of entities
  \end{itemize}

\end{frame}

\begin{frame}[fragile]

\frametitle{Truth in a model, version 1}

\begin{exampleblock}{N-ary Relations}
\begin{verbatim}
    ('boy', set(['b1', 'b2'])),
    ('chase', set([('b1', 'g1'), ('b2', 'g1'), ('g1', 'd1'), ('g2', 'd2')])),
\end{verbatim}
  \end{exampleblock}
\end{frame}

\begin{frame}

  \frametitle{Truth in a model, version 2}

  \begin{itemize}
  \item A model is a pair $\langle D, V\rangle$
  \item Individual terms $\mapsto$ entities
  \item 1-place predicates $\mapsto$ mappings from entities to truth values
  \item 2-place predicates (relations) $\mapsto$ mappings from
    entities to the meanings of 1-place predicates
  \end{itemize}
\end{frame}

\begin{frame}[fragile]

\frametitle{Truth in a model, version 2}

\begin{exampleblock}{Characteristic Functions}
\begin{verbatim}
 'boy': {'b1': True, 'b2': True},
 'chase': {'d1': {'g1': True},
           'd2': {'g2': True},
           'g1': {'b1': True, 'b2': True}},
\end{verbatim}
  \end{exampleblock}
\end{frame}



\end{document}
