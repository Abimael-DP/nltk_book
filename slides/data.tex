\documentclass{beamer}             % for slides
% \documentclass[handout]{beamer}    % for handout
\mode<presentation>
{
  \usetheme{Pittsburgh}
  \setbeamercovered{transparent}
  \beamerdefaultoverlayspecification{<+->}
}

\mode<handout>
{
  \usetheme{default}
  \usecolortheme{default}
  \useoutertheme{default}
  \usepackage{pgf}
  \usepackage{pgfpages}
  \pgfpagesuselayout{4 on 1}[a4paper,landscape,scale=0.9]
  \setjobnamebeamerversion{handout.beamer}
}

\mode<article>
{
  \usepackage{fullpage}
  \usepackage{pgf}
  \usepackage{hyperref}
  \setjobnamebeamerversion{notes.beamer}
}

\usepackage[english]{babel}
\usepackage[latin1]{inputenc}
\usepackage{times}
\usepackage[T1]{fontenc}

\date{\today}

\subject{Natural Language Toolkit}



\title{Linguistic Data Management}

\author{Steven Bird}
\institute{
  University of Melbourne, AUSTRALIA
}

%%%%%%%%%%%%%%%%%%%%%%%%%%%%%%%%%%%%%%%%%%%%%%%%%%%%%%%%%%%%%%%%%%%%%%%%%%%%%%%%%%
%%%%%%%%%%%%%%%%%%%%%%%%%%%%%%%%%%%%%%%%%%%%%%%%%%%%%%%%%%%%%%%%%%%%%%%%%%%%%%%%%%

\begin{document}

\begin{frame}
  \titlepage
\end{frame}

%%%%%%%%%%%%%%%%%%%%%%%%%%%%%%%%%%%%%%%%%%%%%%%%%%%%%%%%%%%%%%%%%%%%%%%%%%%%%%%%%%

\begin{frame}
  \frametitle{Introduction}
  \begin{itemize}
  \item language resources, types, proliferation
  \item role in NLP, CL
  \item enablers: storage/XML/Unicode; digital publication; resource catalogues
  \item obstacles: discovery, access, format, tool
  \item data types: texts and lexicons
  \item useful ways to access data using Python: csv, html, xml
  \end{itemize}

\end{frame}

\begin{frame}
\frametitle{Linguistic Databases}
\begin{itemize}
\item Field linguistics
\item Corpora
\item Reference Corpus
\end{itemize}
\end{frame}

\pgfdeclareimage[width=8cm]{datatypes}{../images/datatypes}
\begin{frame}
\frametitle{Fundamental Data Types}
  \pgfuseimage{datatypes}
\end{frame}

\begin{frame}
\frametitle{Example: TIMIT}
\begin{itemize}
\item TI (Texas Instruments) + MIT
\item balance
\item sentence selection
\item layers of annotation
\item speaker demographics, lexicon
\item combination of time-series and record-structured data
\item programs for speech corpus
\end{itemize}
\end{frame}

\pgfdeclareimage[width=0.7\textwidth]{timit}{../images/timit}
\begin{frame}
\frametitle{Example: TIMIT}
  \pgfuseimage{timit}
\end{frame}

\pgfdeclareimage[width=\textwidth]{timit-structure}{../images/timit-structure}
\begin{frame}
\frametitle{Example: TIMIT}
  \pgfuseimage{timit-structure}
\end{frame}

\begin{frame}[fragile]
\frametitle{Example: TIMIT}
\scriptsize
\begin{verbatim}
    >>> phonetic = list(corpus.timit.phonetic())
    >>> phonetic[0]
    ['h#', 'sh', 'iy', 'hv', 'ae', 'dcl', 'y', 'ix', 'dcl', 'd', 'aa', 'kcl',
    's', 'ux', 'tcl', 'en', 'gcl', 'g', 'r', 'iy', 's', 'iy', 'w', 'aa',
    'sh', 'epi', 'w', 'aa', 'dx', 'ax', 'q', 'ao', 'l', 'y', 'ih', 'ax', 'h#']
    >>> tokenized = list(corpus.timit.tokenized(offset=True))
    >>> tokenized[0]
    [('she', 7812, 10610), ('had', 10610, 14496), ('your', 14496, 15791),
    ('dark', 15791, 20720), ('suit', 20720, 25647), ('in', 25647, 26906),
    ('greasy', 26906, 32668), ('wash', 32668, 37890), ('water', 38531, 42417),
    ('all', 43091, 46052), ('year', 46052, 50522)]
\end{verbatim}
\end{frame}

\begin{frame}[fragile]
\frametitle{Example: TIMIT}
\scriptsize
\begin{verbatim}
    >>> timitdict = corpus.timit.dictionary()
    >>> timitdict['greasy'] + timitdict['wash'] + timitdict['water']
    ['g', 'r', 'iy1', 's', 'iy', 'w', 'ao1', 'sh', 'w', 'ao1', 't', 'axr']
    >>> phonetic[0][17:30]
    ['g', 'r', 'iy', 's', 'iy', 'w', 'aa', 'sh', 'epi', 'w', 'aa', 'dx', 'ax']

    >>> spkrinfo = corpus.timit.spkrinfo()
    >>> spkrinfo['dr1-fvmh0']
    {'use': 'TRN', 'birthdate': '01/08/60', 'sex': 'F', 'race': 'WHT',
    'ht': '5\'05"', 'edu': 'BS', 'recdate': '03/11/86', 'dr': '1',
    'id': 'VMH0', 'comments': 'BEST NEW ENGLAND ACCENT SO FAR'}
\end{verbatim}
\end{frame}

\section{Creating Data}

\begin{frame}
\frametitle{Lifecycle}

\begin{itemize}
\item create
\item annotate texts
\item refine lexicon
\item organize structure
\item publish
\end{itemize}
\end{frame}

\pgfdeclareimage[width=\textwidth]{evolution}{../images/evolution}
\begin{frame}
\frametitle{Evolution}
  \pgfuseimage{evolution}
\end{frame}

\subsection{Primary Data}

\begin{frame}
\frametitle{Creating Data: Primary Data}

\begin{itemize}
\item spiders
\item recording
\item texts
\end{itemize}
\end{frame}

\begin{frame}[fragile]
\frametitle{Data Cleansing: Accessing Spreadsheets}
\scriptsize
\begin{verbatim}
dict.csv:
"sleep","sli:p","v.i","a condition of body and mind ..."
"walk","wo:k","v.intr","progress by lifting and setting down each foot ..."
"wake","weik","intrans","cease to sleep"

    >>> import csv
    >>> file = open("dict.csv", "rb")
    >>> for row in csv.reader(file):
    ...     print row
    ['sleep', 'sli:p', 'v.i', 'a condition of body and mind ...']
    ['walk', 'wo:k', 'v.intr', 'progress by lifting and setting down each foot ...']
    ['wake', 'weik', 'intrans', 'cease to sleep']
\end{verbatim}
\end{frame}

\begin{frame}[fragile]
\frametitle{Data Cleansing: Validation}
\small
\begin{verbatim}
    def undefined_words(csv_file):
        import csv
        lexemes = set()
        defn_words = set()
        for row in csv.reader(open(csv_file)):
            lexeme, pron, pos, defn = row
            lexemes.add(lexeme)
            defn_words.union(defn.split())
        return sorted(defn_words.difference(lexemes))
        
    >>> print undefined_words("dict.csv")
    ['...', 'a', 'and', 'body', 'by', 'cease',
     'condition', 'down', 'each', 'foot',
     'lifting', 'mind', 'of', 'progress',
     'setting', 'to']
\end{verbatim}
\end{frame}

\begin{frame}[fragile]
\frametitle{Data Cleansing: Accessing Web Text}

\scriptsize
\begin{verbatim}
>>> import urllib, nltk
>>> html = urllib.urlopen('http://en.wikipedia.org/').read()
>>> text = nltk.clean_html(html)
>>> text.split()
['Wikimedia', 'Error', 'WIKIMEDIA', 'FOUNDATION', 'Fout', 'Fel',
'Fallo', '\xe9\x94\x99\xe8\xaf\xaf', '\xe9\x8c\xaf\xe8\xaa\xa4',
'Erreur', 'Error', 'Fehler', '\xe3\x82\xa8\xe3\x83\xa9\xe3\x83\xbc',
'B\xc5\x82\xc4\x85d', 'Errore', 'Erro', 'Chyba', 'EnglishThe',
'Wikimedia', 'Foundation', 'servers', 'are', 'currently',
'experiencing', 'technical', 'difficulties.The', 'problem', 'is',
'most', 'likely', 'temporary', 'and', 'will', 'hopefully', 'be',
'fixed', 'soon.', 'Please', 'check', 'back', 'in', 'a', 'few',
'minutes.For', 'further', 'information,', 'you', 'can', 'visit',
'the', 'wikipedia', 'channel', 'on', 'the', 'Freenode', 'IRC', ...
\end{verbatim}
\end{frame}

\subsection{Annotation}

\begin{frame}[fragile]
\frametitle{Creating Data: Annotation}

\begin{itemize}
\item linguistic annotation
\item Tools: \verb|http://www.exmaralda.org/annotation/|
\end{itemize}
\end{frame}

\pgfdeclareimage[width=\textwidth]{windowdiff}{../images/windowdiff}
\begin{frame}[fragile]
\frametitle{Creating Data: Inter-Annotator Agreement}
  \begin{itemize}
  \item Kappa statistic
  \item Windowdiff
  \end{itemize}
  \pgfuseimage{windowdiff}
\end{frame}

\subsection{Lexicons}

\begin{frame}
\frametitle{Processing Toolbox Data}

\begin{itemize}
\item single most popular tool for managing linguistic field data
\item many kinds of validation and formatting not supported by Toolbox software
\item each file is a collection of \textit{entries} (aka
  \textit{records})
\item each entry is made up of one or more \textit{fields}
\item we can apply our programming methods, including chunking and parsing
\end{itemize}
\end{frame}

\begin{frame}[fragile]
\frametitle{Toolbox Example}
\small

\begin{verbatim}
\lx kaa
\ps N.M
\cl isi
\ge cooking banana
\gp banana bilong kukim
\sf FLORA
\dt 12/Feb/2005
\ex Taeavi iria kaa isi kovopaueva kaparapasia.
\xp Taeavi i bin planim gaden banana bilong kukim tasol long paia.
\xe Taeavi planted banana in order to cook it.
\end{verbatim}
\end{frame}

\begin{frame}[fragile]
\frametitle{Accessing Toolbox Data}

\begin{itemize}
\item scan the file, convert into tree object
\item preserves order of fields, gives array and XPath-style access
\end{itemize}

\scriptsize
\begin{verbatim}
>>> from nltk.corpus import toolbox
>>> lexicon = toolbox.parse_corpus('rotokas.dic', key='lx')
\end{verbatim}
\end{frame}

\begin{frame}[fragile]
\frametitle{Accessing with Indexes}

\begin{verbatim}
>>> lexicon[3][0] 
<Element lx at 77bd28>
>>> lexicon[3][0].tag
'lx'
>>> lexicon[3][0].text
'kaa'
\end{verbatim}
\end{frame}

\begin{frame}[fragile]
\frametitle{Accessing with Indexes (cont)}

\scriptsize
\begin{verbatim}
>>> toolbox.to_sfm_string(lexicon[3])
\lx kaa
\ps N.M
\cl isi
\ge cooking banana
\gp banana bilong kukim
\sf FLORA
\dt 12/Feb/2005
\ex Taeavi iria kaa isi kovopaueva kaparapasia.
\xp Taeavi i bin planim gaden banana bilong kukim tasol long paia.
\xe Taeavi planted banana in order to cook it.
\end{verbatim}
\end{frame}

\begin{frame}[fragile]
\frametitle{Accessing with Paths}

\scriptsize
\begin{verbatim}
>>> [lexeme.text.lower() for lexeme in lexicon.findall('record/lx')]
['kaa', 'kaa', 'kaa', 'kaakaaro', 'kaakaaviko', 'kaakaavo', 'kaakaoko',
'kaakasi', 'kaakau', 'kaakauko', 'kaakito', 'kaakuupato', ..., 'kuvuto']
\end{verbatim}

\small
\begin{itemize}
\item lexicon is a series of \texttt{record} objects
\item each contains field objects, such as \texttt{lx} and \texttt{ps}
\item address all the lexemes: \texttt{record/lx}
\end{itemize}

\end{frame}

\begin{frame}
\frametitle{Data Cleansing: Toolbox}

\begin{itemize}
\item parsing (Listing 4)
\item chunking (Listing 5)
\item adding missing fields (next)
\end{itemize}
\end{frame}

\begin{frame}[fragile]
\frametitle{Adding New Fields}
\small

\begin{itemize}
\item Example: add CV field
\item Aside: utility function to do CV template
\end{itemize}

\begin{verbatim}
>>> import re
>>> def cv(s):
...     s = s.lower()
...     s = re.sub(r'[^a-z]',     r'_', s)
...     s = re.sub(r'[aeiou]',    r'V', s)
...     s = re.sub(r'[^V_]',      r'C', s)
...     return (s)
\end{verbatim}
\end{frame}

\begin{frame}[fragile]
\frametitle{Adding New Fields (cont)}
\small

\begin{verbatim}
>>> from nltk.etree.ElementTree import SubElement
>>> for entry in lexicon:
...     for field in entry:
...	    if field.tag == 'lx':
...         cv_field = SubElement(entry, 'cv')
...         cv_field.text = cv(field.text)
\end{verbatim}
\end{frame}

\begin{frame}[fragile]
\frametitle{Adding New Fields (cont)}
\scriptsize
\begin{verbatim}
>>> toolbox.to_sfm_string(lexicon[50])
\lx kaeviro
\cv CVVCVCV
\ps V.A
\ge lift off
\ge take off
\gp go antap
\nt used to describe action of plane
\dt 12/Feb/2005
\ex Pita kaeviroroe kepa kekesia oa vuripierevo kiuvu.
\xp Pita i go antap na lukim haus win i bagarapim.
\xe Peter went to look at the house that the wind destroyed.
\end{verbatim}
\end{frame}

\begin{frame}[fragile]
\frametitle{Generating HTML Tables from Toolbox Data}

\scriptsize
\begin{verbatim}
>>> html = "<table>\n"
>>> for entry in lexicon[70:80]:
...     lx = entry.findtext('lx')
...     ps = entry.findtext('ps')
...     ge = entry.findtext('ge')
...     html += "  <tr><td>%s</td><td>%s</td><td>%s</td></tr>\n" %
...       (lx, ps, ge)
>>> html += "</table>"
>>> print html
\end{verbatim}

\tiny

\begin{verbatim}
<table>
  <tr><td>kakapikoto</td><td>N.N2</td><td>newborn baby</td></tr>
  <tr><td>kakapu</td><td>V.B</td><td>place in sling for purpose of carrying</td></tr>
  <tr><td>kakapua</td><td>N.N</td><td>sling for lifting</td></tr>
  <tr><td>kakara</td><td>N.N</td><td>bracelet</td></tr>
  <tr><td>Kakarapaia</td><td>N.PN</td><td>village name</td></tr>
  <tr><td>kakarau</td><td>N.F</td><td>stingray</td></tr>
  <tr><td>Kakarera</td><td>N.PN</td><td>name</td></tr>
  <tr><td>Kakareraia</td><td>N.???</td><td>name</td></tr>
  <tr><td>kakata</td><td>N.F</td><td>cockatoo</td></tr>
  <tr><td>kakate</td><td>N.F</td><td>bamboo tube for water</td></tr>
</table>
\end{verbatim}
\end{frame}

\begin{frame}[fragile]
\frametitle{Generating XML}

\small
\begin{verbatim}
>>> import sys
>>> from nltk.etree.ElementTree import ElementTree
>>> tree = ElementTree(lexicon[3])
>>> tree.write(sys.stdout)
\end{verbatim}

\scriptsize
\begin{verbatim}
<record>
  <lx>kaakaaro</lx>
  <ps>N.N</ps>
  <ge>mixtures</ge>
  <gp>???</gp>
  <eng>mixtures</eng>
  <eng>charm used to keep married men and women youthful and attractive</eng>
  <cmt>Check vowel length. Is it kaakaaro or kaakaro?</cmt>
  <dt>14/May/2005</dt>
  <ex>Kaakaroto ira purapaiveira aue iava opita, voeao-pa airepa oraouirara, ra va aiopaive.</ex>
  <xp>Kokonas ol i save wokim long ol kain samting bilong ol nupela marit, bai ol i ken kaikai.</xp>
  <xe>Mixtures are made from coconut, ???.</xe>
</record>
\end{verbatim}
\end{frame}

\begin{frame}[fragile]
\frametitle{Analysis: Reduplication}
\scriptsize
\begin{itemize}
\item create a table of lexemes and their glosses

\begin{verbatim}
>>> lexgloss = {}
>>> for entry in lexicon:
...     lx = entry.findtext('lx')
...     if lx and entry.findtext('ps')[0] == 'V':
...         lexgloss[lx] = entry.findtext('ge')
\end{verbatim}

\item For each lexeme, check if the lexicon contains the reduplicated form:

\begin{verbatim}
>>> for lex in lexgloss:
...     if lex+lex in lexgloss:
...         print "%s (%s); %s (%s)" % (lex, lexgloss[lex], lex+lex, lexgloss[lex+lex])
\end{verbatim}
\end{itemize}
\end{frame}

\begin{frame}[fragile]
\frametitle{Reduplication (cont)}
\scriptsize
\begin{verbatim}
kuvu (fill.up); kuvukuvu (stamp the ground)
kitu (save); kitukitu (scrub clothes)
kopa (ingest); kopakopa (gulp.down)
kasi (burn); kasikasi (angry)
koi (high pitched sound); koikoi (groan with pain)
kee (chip); keekee (shattered)
kauo (jump); kauokauo (jump up and down)
kea (deceived); keakea (lie)
kove (drop); kovekove (drip repeatedly)
kape (unable to meet); kapekape (grip with arms not meeting)
kapo (fasten.cover.strip); kapokapo (fasten.cover.strips)
koa (skin); koakoa (remove the skin)
kipu (paint); kipukipu (rub.on)
koe (spoon out a solid); koekoe (spoon out)
kovo (work); kovokovo (surround)
kiru (have sore near mouth); kirukiru (crisp)
kotu (bite); kotukotu (grind teeth together)
kavo (collect); kavokavo (work black magic)
...
\end{verbatim}
\end{frame}

\subsection{Analysis: Complex Search Criteria}

\begin{frame}[fragile]
\frametitle{Complex Search Criteria}
\scriptsize
\begin{verbatim}
>>> from nltk import tokenize, FreqDist
>>> fd = FreqDist()
>>> lexemes = [lexeme.text.lower() for lexeme in lexicon.findall('record/lx')]
>>> for lex in lexemes:
...     for syl in tokenize.regexp(lex, pattern=r'[^aeiou][aeiou]'):
...         fd.inc(syl)
\end{verbatim}

\normalsize
\begin{itemize}
\item for phonological description, identify segments, alternations,
  syllable canon...
\item what syllable types occur in lexemes (MSC, conspiracies)?
\end{itemize}

\end{frame}

\begin{frame}[fragile]
\frametitle{Analysis: Complex Search Criteria (cont)}

\begin{itemize}
\item Tabulate the results:

\scriptsize
\begin{verbatim}
>>> for vowel in 'aeiou':
...     for cons in 'ptkvsr':
...          print '%s%s:%4d ' %
...            (cons, vowel, fd.count(cons+vowel)),
...     print
pa:  84  ta:  43  ka: 414  va:  87  sa:   0  ra: 185 
pe:  32  te:   8  ke: 139  ve:  25  se:   1  re:  62 
pi:  97  ti:   0  ki:  88  vi:  96  si:  95  ri:  83 
po:  31  to: 140  ko: 403  vo:  42  so:   3  ro:  86 
pu:  49  tu:  35  ku: 169  vu:  44  su:   1  ru:  72 
\end{verbatim}

\small
\item NB \texttt{t} and \texttt{s} columns
\item \texttt{ti} not attested, while \texttt{si} is frequent: palatalization?
\item which lexeme contains \texttt{su}?  \textit{kasuari}
\end{itemize}
\end{frame}


\begin{frame}
\frametitle{Analysis: Finding Minimal Sets}
\begin{itemize}
\item E.g. mace vs maze, face vs faze
\item minimal set parameters: context, target, display
\vfil
\small

\begin{tabular}{l|lll}
\textbf{Minimal Set} &     \textbf{Context} &  \textbf{Target} & \textbf{Display} \\ \hline
\textit{bib, bid, big} &   first two letters&  third letter    & word\\
\textit{deal (N), deal (V)} & whole word    &  pos             & word (pos)
\end{tabular}
\end{itemize}
\end{frame}

\begin{frame}[fragile]
\frametitle{Finding Minimal Sets: Example 1}
\scriptsize

\begin{verbatim}
>>> from nltk import MinimalSet
>>> pos = 1
>>> ms = MinimalSet((lex[:pos] + '_' + lex[pos+1:], lex[pos], lex)
...                 for lex in lexemes if len(lex) == 4)
>>> for context in ms.contexts(3):
...     print context + ':',
...     for target in ms.targets():
...         print "%-4s" % ms.display(context, target, "-"),
...     print
k_si: kasi -    kesi -    kosi
k_ru: karu kiru keru kuru koru
k_pu: kapu kipu -    -    kopu
k_ro: karo kiro -    -    koro
k_ri: kari kiri keri kuri kori
k_pa: kapa -    kepa -    kopa
k_ra: kara kira kera -    kora
k_ku: kaku -    -    kuku koku
k_ki: kaki kiki -    -    koki
\end{verbatim}
\end{frame}

\begin{frame}[fragile]
\frametitle{Finding Minimal Sets: Example 2}
\scriptsize

\begin{verbatim}
>>> entries = [(e.findtext('lx'), e.findtext('ps'), e.findtext('ge'))
...               for e in lexicon
...               if e.findtext('lx') and e.findtext('ps') and e.findtext('ge')]
>>> ms = MinimalSet((lx, ps[0], "%s (%s)" % (ps[0], ge))
...          for (lx, ps, ge) in entries)
>>> for context in ms.contexts()[:10]:
...     print "%10s:" % context, "; ".join(ms.display_all(context))
  kokovara: N (unripe coconut); V (unripe)
     kapua: N (sore); V (have sores)
      koie: N (pig); V (get pig to eat)
      kovo: C (garden); N (garden); V (work)
    kavori: N (crayfish); V (collect crayfish or lobster)
    korita: N (cutlet?); V (dissect meat)
      keru: N (bone); V (harden like bone)
  kirokiro: N (bush used for sorcery); V (write)
    kaapie: N (hook); V (snag)
       kou: C (heap); V (lay egg)
\end{verbatim}
\end{frame}


\section{Publishing}

\begin{frame}
\frametitle{Publishing}

\begin{itemize}
\item metadata: DC, OLAC
\item repositories
\item search
\item demo
\end{itemize}

\end{frame}

\end{document}
