\documentclass{beamer}             % for slides
% \documentclass[handout]{beamer}    % for handout
\mode<presentation>
{
  \usetheme{Pittsburgh}
  \setbeamercovered{transparent}
  \beamerdefaultoverlayspecification{<+->}
}

\mode<handout>
{
  \usetheme{default}
  \usecolortheme{default}
  \useoutertheme{default}
  \usepackage{pgf}
  \usepackage{pgfpages}
  \pgfpagesuselayout{4 on 1}[a4paper,landscape,scale=0.9]
  \setjobnamebeamerversion{handout.beamer}
}

\mode<article>
{
  \usepackage{fullpage}
  \usepackage{pgf}
  \usepackage{hyperref}
  \setjobnamebeamerversion{notes.beamer}
}

\usepackage[english]{babel}
\usepackage[latin1]{inputenc}
\usepackage{times}
\usepackage[T1]{fontenc}

\date{\today}

\subject{Natural Language Toolkit}



\title{Advanced Programming in Python}

\author{Steven Bird \and Ewan Klein \and Edward Loper}
\institute{
  University of Melbourne, AUSTRALIA
  \and
  University of Edinburgh, UK
  \and
  University of Pennsylvania, USA
}

%%%%%%%%%%%%%%%%%%%%%%%%%%%%%%%%%%%%%%%%%%%%%%%%%%%%%%%%%%%%%%%%%%%%%%%%%%%%%%%%%%
%%%%%%%%%%%%%%%%%%%%%%%%%%%%%%%%%%%%%%%%%%%%%%%%%%%%%%%%%%%%%%%%%%%%%%%%%%%%%%%%%%

\begin{document}

\begin{frame}
  \titlepage
\end{frame}

\section{Regular Expressions}

%%%%%%%%%%%%%%%%%%%%%%%%%%%%%%%%%%%%%%%%%%%%%%%%%%%%%%%%%%%%%%%%%%%%%%%%%%%%%%%%%%


\section{Program Development}

\begin{frame}
\frametitle{Program Development}
\begin{itemize}
\item High-level Design
\item Algorithm Design
\item Structured Programming
\item Debugging
\item Performance Tuning
\end{itemize}
\end{frame}

\subsection{Defining Functions and Modules}

\begin{frame}[fragile]
\frametitle{Defining Functions}
\scriptsize
\begin{itemize}
\item part of a program needs to be used several times over
\item e.g. form plural of singular noun, to be done in several places
  in a program
\item localize this work inside a \textit{function}
\item also helps readability, reusability

\begin{verbatim}
  >>> def plural(word):
  ...     if word[-1] == 'y':
  ...         return word[:-1] + 'ies'
  ...     elif word[-1] in 'sx':
  ...         return word + 'es'
  ...     elif word[-2:] in ['sh', 'ch']:
  ...         return word + 'es'
  ...     elif word[-2:] == 'an':
  ...         return word[:-2] + 'en'
  ...     return word + 's'
  >>> plural('fairy')
  'fairies'
  >>> plural('woman')
  'women'
\end{verbatim}
\end{itemize}
\end{frame}

\begin{frame}[fragile]
  \frametitle{Defining Modules}

\begin{verbatim}
def parsed(files):
    for file in files:
        path = os.path.join(get_basedir(), "treebank", file)
        s = open(path).read()
        for t in tokenize.blankline(s):
            yield tree.bracket_parse(t)
\end{verbatim}
\end{frame}

\subsection{Algorithm Design}

\begin{frame}
  \frametitle{Algorithm Design}
  \small

  \begin{itemize}
  \item \textit{algorithm}: a "recipe" for solving a problem
  \item e.g. to multiply 16 by 12 we might use any of the following methods:

    \begin{enumerate}
    \item Add 16 to itself 12 times over
    \item Perform "long multiplication", starting with the least-significant
      digits of both numbers
    \item Look up a multiplication table
    \item Repeatedly halve the first number and double the second,
      16*12 = 8*24 = 4*48 = 2*96 = 192
    \item Do 10*12 to get 120, then add 6*12
    \end{enumerate}

  \item computation time, intermediate storage
  \item brute-force, divide-and-conquer, dynamic programming, greedy search
  \item Textbook: Levitin, Anany (2003) \textit{Introduction to the Design and
      Analysis of Algorithms}
  \end{itemize}
\end{frame}

\begin{frame}[fragile]
\frametitle{Sorting Algorithms}

\begin{itemize}
\item Many algorithms for sorting
\item Illustrates the difference in algorithm complexity

\begin{verbatim}
  >>> from random import shuffle
  >>> from nltk_lite.misc import sort
  >>> a = range(1000)
  >>> shuffle(a); sort.bubble(a)
  250918
  >>> shuffle(a); sort.merge(a)
  6175
  >>> shuffle(a); sort.quick(a)
  2378
\end{verbatim}
\end{itemize}
\end{frame}

\subsection{Top-Down Design}

\begin{frame}
\frametitle{Top-Down Design}
\begin{itemize}
\item Break down high-level task into manageable components
\item Build and test each component
\item Assemble them into a complex system
\item Example: identify adjective-noun collocations in text
  \begin{enumerate}
  \item read in the corpus
  \item count events (each adjective, noun, adj-noun combination)
  \item compute $\chi^2$ statistics
  \item sort adj-noun pairs in decreasing order
  \item output $n$-most significant collocations
  \end{enumerate}
\item what are the inputs and outputs of each step (i.e. \textit{interfaces})
\end{itemize}
\end{frame}

\begin{frame}[fragile]
\frametitle{Top-Down Design (cont)}

\begin{enumerate}
\item Define top-level function:

\begin{verbatim}
def colloc(corpus_name, num):
    corpus = load_corpus(corpus_name)
    a_counts, n_counts, a_n_counts = count(corpus, 'JJ', 'NN')
    a_n_scores = chisq(a_counts, n_counts, a_n_counts)
    a_n_ranked = sort_by_score(a_n_scores)
    return a_n_ranked[:num]
\end{verbatim}

\item Iteratively define lower-level functions, e.g. \texttt{chisq()}

\item Bottom-up testing
\begin{itemize}
\item standard test cases for components, known output
\item only assemble once each component is known to work correctly
  on the test cases
\end{itemize}
\end{enumerate}
\end{frame}

\subsection{Debugging}

\begin{frame}
  \frametitle{Debugging}

  \begin{itemize}
  \item What's in the name:
    \begin{itemize}
    \item problems are small relative to their impact
    \item hard-to-find
    \item seem to take on a life of their own as the programmer tries to
      hunt them down
    \end{itemize}
  \item steps:
    \begin{itemize}
    \item isolate the problem
    \item syntax error: see error messages, linked to line numbers
    \item run-time error: \textit{stack-trace}
    \item add diagnostic print statements
    \item display values of variables just before problem
    \item Python debugger: \texttt{pdb}
    \end{itemize}
  \end{itemize}
\end{frame}

\subsection{Next Steps...}

\begin{frame}
\frametitle{Next Steps: Learning to Program}
\begin{itemize}
\item experimental
\item Python's \texttt{help} and \texttt{dir} commands
\item tutorial exercises
\end{itemize}
\end{frame}

