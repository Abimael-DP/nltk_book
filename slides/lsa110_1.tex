\documentclass{beamer}             % for slides
% \documentclass[handout]{beamer}    % for handout
\mode<presentation>
{
  \usetheme{Pittsburgh}
  \setbeamercovered{transparent}
  \beamerdefaultoverlayspecification{<+->}
}

\mode<handout>
{
  \usetheme{default}
  \usecolortheme{default}
  \useoutertheme{default}
  \usepackage{pgf}
  \usepackage{pgfpages}
  \pgfpagesuselayout{4 on 1}[a4paper,landscape,scale=0.9]
  \setjobnamebeamerversion{handout.beamer}
}

\mode<article>
{
  \usepackage{fullpage}
  \usepackage{pgf}
  \usepackage{hyperref}
  \setjobnamebeamerversion{notes.beamer}
}

\usepackage[english]{babel}
\usepackage[latin1]{inputenc}
\usepackage{times}
\usepackage[T1]{fontenc}

\date{\today}

\subject{Natural Language Toolkit}



\title{Python Programming for Linguists\\LSA 100 Presession}

% \author{Steven Bird \and Ewan Klein \and Edward Loper}
% \institute{
%   University of Melbourne, AUSTRALIA
%   \and
%   University of Edinburgh, UK
%   \and
%   University of Pennsylvania, USA
% }

%%%%%%%%%%%%%%%%%%%%%%%%%%%%%%%%%%%%%%%%%%%%%%%%%%%%%%%%%%%%%%%%%%%%%%%%%%%%%%%%%%
%%%%%%%%%%%%%%%%%%%%%%%%%%%%%%%%%%%%%%%%%%%%%%%%%%%%%%%%%%%%%%%%%%%%%%%%%%%%%%%%%%

\begin{document}


%%%%%%%%%%%%%%%%%%%%%%%%%%%%%%%%%%%%%%%%%%%%%%%%%%%%%%%%%%%%%%%%%%%%%%%%%%%%%%%%%%

\begin{frame}
  \titlepage
\end{frame}



\begin{frame}
  \frametitle{Introduction}
  \begin{itemize}
    \item Who we are
    \item Python and NLTK
    \item Materials and Resources
    \item Goals
    \item Syllabus
  \end{itemize}
\end{frame}

\begin{frame}
  \frametitle{Who we are}

Instructors:
  \begin{itemize}
    \item Steven Bird
    \item Ewan Klein
    \item Edward Loper (here tomorrow)
  \end{itemize}

TAs:
  \begin{itemize}
    \item David Hall
    \item Yaron Greif
    \item Yun-Hsuan Sung
%     \item Jette Viethen
  \end{itemize} 

\end{frame}


\begin{frame}
  \frametitle{Python and NLTK}
  \begin{itemize}
    \item Pre-session for \textit{Introduction to
        Computational Linguistics} (LSA 325)
    \item First steps in using Python and Natural Language Toolkit (NLTK)
    \item Why Python?
      \begin{itemize}
      \item designed to be easy to learn;
      \item good for processing linguistic data;
      \item good for interactive experiments.
      \end{itemize}
    \item Many online tutorials (see \url{www.python.org})
  \end{itemize}
\end{frame}


\begin{frame}
  \frametitle{Materials and Resources}
  \begin{itemize}
  \item Chapter 2, \textit{Programming Fundamentals and Python} in the NLTK Book
    (\url{http://nltk.org/index.php/Book})
    \item \textbf{Su07-LSA-110} page on \url{http://coursework.stanford.edu}
    \item Main NLTK page: \url{http://nltk.org}
      \begin{itemize}
        \item Chatroom
        \item \texttt{nltk-users} Mailing List
      \end{itemize}

   \end{itemize}
\end{frame}

\begin{frame}
  \frametitle{Audience and Goals}
  \begin{itemize}
  \item We are assuming you have not done programming before.
  \item So, getting you to a point where:
    \begin{itemize}
    \item you have got some confidence in using basic Python commands;
    \item you can use Python for carrying out simple operations on text;
    \item you can do all the easy and intermediate exercises in
      Chapter 2;
    \item you have found out where to get more information (fellow
      students, the web, textbooks)
    \end{itemize}
   \end{itemize}
\end{frame}

\begin{frame}
  \frametitle{Syllabus}
  \begin{description}
  \item[Class 1] Manipulating strings, lists and other sequences.
  \item[Class 2] Conditionals, dictionaries, functions and regular
    expressions.
  \item[Class 3] Preview of NLTK chapters on Words and Tagging
   \end{description}
\end{frame}

\begin{frame}
  \frametitle{Almost there \ldots}
  \begin{itemize}
  \item Installation CDs
  \item Today: at least Python
  \item Tomorrow: full NLTK installation
  \item Homework: catch up on exercises and reading
   \end{itemize}
\end{frame}




\end{document}
