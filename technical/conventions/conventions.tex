% Natural Language Toolkit Technical Report:
% Conventions
%
% Copyright (C) 2001 University of Pennsylvania
% Author: Edward Loper <edloper@gradient.cis.upenn.edu>
% URL: <http://nltk.sf.net>
% For license information, see LICENSE.TXT
%
% $Id$

\documentclass{article}
\usepackage{fullpage}
\usepackage{boxedminipage}

\begin{document}
\title{Conventions}
\author{Edward Loper}
\maketitle

%################################################################
%#  OVERVIEW
%################################################################
\section{Overview}


\section{Immutable vs Mutable Variables}

  Basic data types should generally be immutable.

\section{Nomenclature}

  \subsection{Variable Names}

    all lower case, with underscores

  \subsection{Class Names}

    WordInitialCaps

  \subsection{Module Names}

    short, lower case names.

\section{Packages and Modules}


\section{Data Classes}


\section{Interfaces}

  use AssertionError for methods that must be implemented.

  use NotImplementedError for optional methods.

\section{String Representations}

  str is for multiline; repr is single-line, embeddable.  e.g., str
  for production is something like ``A -> B C'' but for embeddable
  (repr), it's more like ``[production: A -> B C]''..  etc.

\section{Task Interfaces & Classes}

  \subsection{Multiple Return Values}

  \subsection{Trace Output}

    trace output is controlled by a single ``trace level'' instance
    variable.

    trace values: 0: no output; 1: less than a page for a typical
    run; 2 or higher: more verbose.

    setting trace output:
      - by default, trace=0
      - constructor takes a trace keyword argument
      - trace(n) method sets trace level; use default value for n.

\end{document}
