% Natural Language Toolkit Technical Report:
% Representations for Printing
%
% Copyright (C) 2001 University of Pennsylvania
% Author: Edward Loper <edloper@gradient.cis.upenn.edu>
% URL: <http://nltk.sf.net>
% For license information, see LICENSE.TXT
%
% $Id$

\documentclass[11pt]{article}
\usepackage{fullpage}

\begin{document}
\title{Representations for Printing\\
\Large Natural Language Processing Toolkit: Technical Report}
\author{Edward Loper}
\maketitle

% One thing that guides peoples intuitions about what an object is is
% how it looks when it prints. If two things look the same when you
% print them, then intuitively they should be the same. I think this may
% have been a source of confusion in the first problem set, where Tokens
% look just like strings when you print them, so users will be tempted
% to do things like use == to test whether a TokenType is equal to a
% string.

%============= INTRODUCTION =============%
\section{Introduction}

  The printed representations of an object guides users' intuitions
  about the content, function, and purpose of the object.  Often, a
  user's first encounter with an object is with its printed
  representation.  Users rely on this representation to let them know
  what type of object it is, and what its important features are.  The
  printed representations of each type of object defined by the NLP
  toolkit must be carefully considered.

%============= GOALS =============%
\section{Goals}

  The following goals guide the design of printed representations.

  \begin{description}

    \item[Compactness] Printed representations should be simple and
    compact.  Simpler representations are easier to read and easier to
    understand, especially when the object is included in the
    representation of a larger data structure, such as a list of
    objects.

    \item[Consistancy] The printed representation of an object should
    be consistant between different environments.  Inconsistant
    representations can produce confusion about object values and
    object types.

    \item[Composition] The representation of an object should indicate
    what data fields it abstractly contains.  In particular, for each
    data field abstractly contained by an object, its representation
    should include some indication that the field is contained within
    it.

    \item[Type Indication] The representation should give some direct
    or indirect indication of its type.  In particular, it should be
    easy to tell the type of an object from its printed
    representation.

    \item[Value Indication] If two objects have the same value, they
    should have the same printed representations; if two objects have
    different values, then they should have different printed
    representations.

    \item[Intuitiveness] Representations should rely on the
    conventional meanings of formats and symbols (e.g., the use of
    ``@'' to denote ``at'').  This will make it easier for users to
    understand what the representations stand for.

    \item[Standard Representations] When possible, established
    standard representations (such as ``dog/NP'' for tagging) should
    be used.

  \end{description}

%============= ISSUES =============%
\section{Issues}

The representations are still under development.  This section
describes some of the issues involved in designing representations.  I
may not use the solutions discussed here.

\subsection{Context-Dependant Representations}

  Representations that show the complete state of an object are useful
  for certain tasks, but they can be overly verbose for complex
  objects, or for lists of objects, making it hard to understand the
  structures shown by the printed representations.  In particular,
  complete representations of complex objects are not \emph{compact}.

  The principle of \emph{composition} requires that the representation
  display each field of an object; but it may be useful to display a
  ``summary'' of some of the fields, instead of a complete
  representation.  For example, a \texttt{Location} whose unit is
  \texttt{"token"} and whose source is \texttt{"foo.txt"} might
  normally be represented as:

  \begin{quote}
    \texttt{@[token 1:token 3]@foo.txt}
  \end{quote}

  But if a large list of tokens is displayed, each with a location,
  then this representation may be overly verbose.  A ``summary'' of
  this location might be:

  \begin{quote}
    \texttt{@[1:3]}
  \end{quote}

  Note that this solution is somewhat contrary to \emph{consistancy}:
  we are using different representations for the same object in
  different environments.  However, if summary representations are
  chosen carefully enough, I don't believe that this will be a
  problem.

\subsubsection{\texttt{str} and \texttt{repr}}

  The \texttt{str} and \texttt{repr} functions could be used to
  implement the difference between full representations and summary
  representations.  Officially, \texttt{str} gives the ``informal
  representation'' of an object and \texttt{repr} gives its ``formal
  representation.''  However, one of the main differences between them
  in practice is that \texttt{str} is used when a single object is
  printed, but \texttt{repr} is used when an object is printed as part
  of a list, tuple, or dictionary.  

  We could extend this distinction to the toolkit's objects, using
  \texttt{str} to encode an object's top-level representation, and
  \texttt{repr} to encode its summary.  The advantage of this approach
  is that it guarantees that each object has exactly one summary
  representation and one full representation.  For some objects, these
  may be the same representation.

\subsection{Refactoring Information}

  Unfortunately, summary representations don't always seem like the
  right tool to make representations more compact.  In a number of
  cases, the representation of a complex object may contain a large
  amount of redundant information.  It may make sense to display this
  information exactly once.  e.g., syntax trees: how to attatch
  locations?

%============= PRINTING REPS =============%
\section{Printing Representations}

Here's a list of how I'm currently representing things:

\newpage
\begin{tabbing}
\tt
\hspace{3in}\=\\
\textbf{$>>>$ print Location(1, 3)}\\
@[1:3]\\
\textbf{$>>>$ print Location(1, 2)}\\
@[1]\\
\textbf{$>>>$ print Location(1, 3, source='foo.txt')}\\
@[1:3]@'foo.txt'\\
\textbf{$>>>$ print Location(1, 3, unit='word')}\\
@[word 1:word 3]\\
\textbf{$>>>$ print Location(1, 2, unit='word', source='foo.txt')}\\
@[word 1:word 2]\\
\\
\textbf{$>>>$ print Token('hi', Location(1, 2))}\\
'hi'@[1]\\\
\textbf{$>>>$ print Token('hi', Location(1, 3))}\\
'hi'@[1:3]\\\
\textbf{$>>>$ print Token('hi', Location(1, 3, unit='word', source='foo.txt'))}\\
'hi'@[1:3] \>\textrm{Loc's unit and source are not shown}\\
\textbf{$>>>$ }\\
\\
\textbf{$>>>$ print Tree('np', 'the', 'boy')}\\
('np': 'the' 'boy') \>\textrm{No commas and presence of ':' distinguises from tuple}\\
\textbf{$>>>$ print Tree('vp', 'likes', Tree('np', 'the', 'boy'))}\\
('vp': 'likes' ('np': 'the' 'boy'))\\
\textbf{$>>>$ print TreeToken('vp', Token('likes',1), Tree('np', Token('John',2)))}\\
('vp': 'likes' ('np': 'John))@[1:3]\\


\end{tabbing}


%============= RANDOM NOTES =============%
\section{Random Notes}

\subsection{Issues}

There are a number of considerations when deciding how to represent
objects:

\begin{itemize}

  \item Users should be able to tell what type of object something is
    by printing it.  In particular:
    \begin{itemize}
      \item It should be possible to distinguish an object from any
        built-in type (string, list, etc)
      \item It should be possible to distinguish each of the different
        classes defined by the toolkit from each other.
    \end{itemize}

    Of course, this is within reason -- a string can look like any
    object, for example.

  \item How much to display?  Representations that show the complete
  state of an object can be useful, but they can be overly verbose for
  large objects, or for lists of objects, making it hard to understand
  the structures shown by the printed representations.

  \item One rep per object, or not?  In particular, should there be a
  different embedded representation?  Difference between str and repr
  as embedding.

  \item Representations should be intuitive

  \item Representations should be simple

  \item Should representations be readable?

  \item Should standard reps be used (eg, treebank)

  \item Should some classes be treated specially (eg, quote strings)

\end{itemize}

\end{document}

