%
% File acl02sub.tex
%
% Contact: acl02-submission@cs.ualberta.ca

\documentclass[11pt]{article}
\usepackage{acl02sub}
\title{Instructions for ACL-02 Submission}
\author{
\begin{tabular}[t]{c}
Eugene Charniak\\
Computer Science Department\\
Brown University\\
Providence, RI, USA, 02912 
\end{tabular}
\hspace{6pt}
\begin{tabular}[t]{c}
Dekang Lin\\
Department of Computing Science\\
University of Alberta\\
Edmonton, Alberta, Canada, T6G 2E8
\end{tabular}
}
\summary{%
  This document is an example of what your submission manuscript to
  ACL-02 should look like.  Authors are asked to conform to the
  directions reported in this document.  See the ACL-02 call for
  papers for further instructions about the submission.}
\paperid{Pxxxx}
\keywords{keyword1, keyword2, keyword3, keyword4, keyword5}
\contact{author of record (for correspondence)}
\conference{submission to other conference(s)}
\date{}

\begin{document}
\makeidpage
\maketitle

\section{Introduction}

This document is a slightly revised version of the instructions
for preparing copies for the final ACL-02 proceedings. The
format here described allows for a graceful transition to the
style required for that publication.

This document has been elaborated from similar documents used for
previous editions of the ACL and EACL annual meetings. Those
versions were written by several people, including John Chen,
Henry S. Thompson and Donald Walker. This document is an example
of what your submission manuscript to ACL-02 should look like.
Authors are asked to conform to the directions reported in this
document.


\section{General Instructions}

As reviewing will be blind, a separate identification page will be
required reporting authors' names, affiliations, and email addresses
(see below for additional information on the identification page).
The paper itself should not include the authors' names and
affiliations. Furthermore, self-references that reveal the authors'
identities, e.g., ``We previously showed \cite{dawidskene79} ...'',
should be avoided. Instead, use citations such as ``Dawid and Skene
\shortcite{dawidskene79} previously showed ...''  In general, authors
should avoid including any information in the body of the paper, in
the acknowledgments and in the reference list, that would identify the
authors themselves or their institutions. Such information can be
added to the final camera-ready version for publication.

The maximum length of a submission manuscript is $8$ pages,
printed single-sided.  Print all text, including section titles
and figures, in two-column format where each column is 7.6cm by
23.5cm (3.0in by 9.25in) and there is a 0.6cm (0.25in) space
between the two columns. Exceptions to the two-column format
include the title at the top of the first page and any full-width
figures or tables. (See the guidelines later regarding formatting
the first page of the submission manuscript.)  Start all pages
directly under the top margin. Text should be centered on each
page. On A4 paper, this roughly means leaving 2.5cm (1in) margins
on left and right sides of each page as well as a 3.0cm (1.2in)
margin on the top of each page. Type single spaced. Indent when
starting a new paragraph. Use standard fonts such as Times Roman
or Computer Modern Roman, with 11 points for text and 14 to 16
points for headings and title.


\subsection{Electronically-available resources}

This file ({\tt acl02sub.tex}) along with the LaTeX style file ({\tt
  acl02sub.sty}) and ACL bibliography style ({\tt acl.bst}) are
available from\\
http://www.cs.ualberta.ca/$\sim$lindek/acl02/style/\\
A
Microsoft Word Style file ({\tt acl02sub.doc}) is also present at the
same URL.  We strongly recommend the use of these style files, that
have been appropriately tailored for the ACL-02 submission.


\subsection{The identification page}

As reviewing will be blind, a separate identification page is required.
The identification page should include the paper title, authors' names,
affiliations, and email addresses, the paper ID code generated upon
paper registration (see the ACL-02 call for papers for instructions
about the electronic registration procedure for submitted papers), up to
5 keywords specifying the subject area, name and address of the contact
author, and a short summary (up to 5 lines).  The identification page
should also specify whether the paper is under consideration for other
conferences.



\subsection{The First Page}

Center the title across both columns.  As already mentioned above,
do not include authors' names and affiliations.  Use the
two-column format only when you begin the abstract.

{\bf Title}: Place the title at the top of the first page, followed by
the paper ID code as generated upon paper registration.  Long title
should be typed on two lines without a blank line
intervening. Approximately, put the title at 2.7cm (1.1in) from the top
of the page, and leave 2.5cm (1in) between the title and the body of the
first page.

{\bf Abstract}: Type the abstract at the beginning of the first column.
The width of the abstract text should be smaller than the width of the
columns for the text in the body of the paper by about 0.7cm (0.3in) on
each side.  Center the word {\bf Abstract} in bold form above the body
of the abstract. The abstract should be no longer than 200 words.

{\bf Text}: Begin typing the main body of the text immediately
after the abstract, observing the two-column format as shown in
this example.


\subsection{Sections}

{\bf Headings}: Type and label section and subsection headings in
the style shown on these pages.  Use numbered sections, in order
to facilitate cross references.

{\bf References}: Follow the ``Guidelines for Formatting
Submissions'' to {\em Computational Linguistics} that appears in
the first issue of each volume, if possible.  That is, citations
within the text appear in parentheses as (Author, year) or, if the
author's name appears in the text itself, as Author (year). Gather
the full set of references together under the heading {\bf
References}; place the section before any {\bf Appendices}, unless
they contain references. Arrange the references alphabetically by
first author, rather than by order of occurrence in the text.
Provide as complete a citation as possible, using a consistent
format, such as the one for CL or the one in the {\em
Publication Manual of the American Psychological Association}
(American Psychological Association, 1983).  Use of full names for
authors rather than initials is preferred.  A list of
abbreviations for common computer science journals can be found in
the ACM {\em Computing Reviews} (Association for Computing
Machinery, 1983).

The LaTeX and BibTeX style files provided roughly fit the APA format,
allowing regular citations (Author, year), short citations for text
that mentions the Author (year), and multiple citations (Author, year;
Author, year; Author, year).

{\bf Appendixes}: Appendixes, if any, directly follow the text and the
references (but see above).  Letter them in sequence and provide an
informative title: {\bf Appendix A. Title of Appendix.}


\subsection{Footnotes}

{\bf Footnotes}: Put footnotes at the bottom of the page. They may
be numbered or referred to by asterisks or other
symbols.\footnote{This is how a footnote should appear.} Footnotes
should be separated from the text by a line.\footnote{Note the
line separating the footnotes from the text.}

\subsection{Graphics}

{\bf Illustrations}: Place figures, tables, and photographs in the paper
near where they are first discussed, rather than at the end, if
possible.  Wide illustrations may run across both columns.

{\bf Captions}: Provide a caption for every illustration; number each one
sequentially in the form:  ``Figure 1. Caption of the Figure.'' ``Table 1.
Caption of the Table.''  Type the captions for figures below the
figures.  Type the captions for tables above the tables.
%check that the .sty does so


\section{Length of Submission}

Eight pages ($8$) is the length limit for ACL-02 submissions. All
illustrations, references, and appendices must be accommodated within
these page limits, also observing the formatting instructions given in
the present document. This will also be the number of pages allocated
for each paper in the ACL-02 proceedings.

Papers that do not conform to the specified length
and formatting requirements are subject to be rejected without
review.


\section{Submission Procedure}


\paragraph{Paper registration:} You must submit a notification of
submission by filling out the form at\\
\centerline{http://www.cs.ualberta.ca/$\sim$lindek/acl02}\\
The authors should fill in the title of the paper, the authors' names,
affiliations, and email addresses, one or two general topic areas, up
to 5 keywords specifying the subject area, and a short summary (up to
200 words).  The authors should also specify whether the paper is
under consideration for other conferences or workshops, and if so,
which ones.
  
Each submission will be assigned an identification number. Please use
it on all correspondence with the program committee.
  
\paragraph{Paper submission:} All papers must be submitted electronically
at the same web address. The first page of your paper must include the
identification number obtained from paper registration. The paper must
be submitted no later than 12 noon Mountain Time (7PM GMT) on Feb. 1
2002. Papers submitted after that time will not be reviewed. Papers
must be in PDF format. The submission web page includes information
about converting different types of documents to PDF. The program
committee will make every attempt to print out your paper
successfully, but cannot take responsibility if they do not. Authors
are strongly encouraged to submit papers 48 hours before the
submission deadline so that unprinting formats can be detected and
corrected by the submission deadline. If for some reason an author is
not able to submit electronically, or if we have discovered in advance
that there is a problem with the PDF file, authors should contact
Dekang Lin (lindek@cs.ualberta.ca) concerning hard-copy submission.


\bibliographystyle{acl}

\bibliography{sample}

%\section*{References}
%
%American Psychological Association.  1983.  {\em Publications Manual}.
%Washington, D.C.: American Psychological Association.
%
%Association for Computing Machinery.  1983.  {\em Computing Reviews}, 24(11):
%503-512.

\end{document}
